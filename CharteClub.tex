\documentclass[12pt,a4paper]{article}
\usepackage[utf8]{inputenc}
\usepackage[francais]{babel}
\usepackage[T1]{fontenc}
\usepackage{graphicx}
\usepackage[left=2cm,right=2cm,top=2cm,bottom=2cm]{geometry}
\usepackage{comment}
\title{Projet club FabLab}
\begin{document}
\section{Introduction}
\section{Mission}
En creant ce club au FabLab nous souaiterions pouvoire permettre au FabLab de devenir plus qu'un simple laboratoire de prototypage dedier aux projets mais un lieux de vie, d'apprentissage et d'epanouissement au coeure de l'ECE. C'est dans cette esprit de tirer le plein potenciel de ce lieux unique que nous vous invitons decouvrir les missions que nous cherchons a affronter. \\
\subsection{Acces elargie aux eleves}
Nous sommes parti du constat qu'il y'avait une demande de la part des eleves de pouvoire utiliser le materiel hors du cadre stricte des projets scolaire. En effet, il existe chez certains eleves une vrais volonter de construire et de donner vie a leurs propres projets. Mais cette volonter se heurter aux limitation d'acces du FabLab tel qu'il est consu actuellement. Sans une structure  encadrente claire il n'est pas possible d'acceuillir les eleves dans les bonnes conditions.\\
\paragraph{Problematique:}
Ce probleme en realiter en cache plusieurs: tout d'abbord il faut veiller a la bonne utilisation des machines et du respect du lieu par les eleves. Ceci suppose donc que en plus de devoir veiller a ce que tout les eleves sache manier les machines il faudrait s'assurer d'avoire sans cesse un responsable present. Un systeme de formation devra donc etre mise en place en parallele d'une forme de controle d'acce la moin restrictive possible.\\
\paragraph{Solution: }
Tout d'abbord il me semble important de rappeller ici que notre but n'est en auqune maniere de restrindre l'acce au FabLab dans le cadre des projets s'incrivant dans le parcour pedagogique de l'ECE. L'acce aux machines et aux materiel leur sera prioritaire et les membres du club seront pres a leur apporter assistance en toutes circonstance leur de leurs permances.\\
Nous allons commencer par distinger 3 groupes distincs d'eleves : les Makers , les Membres et les eleves non-membres.\\
\subparagraph{Makers} Ce sont les membres les plus experimentez, c'est eux qui seront charger de la transmission de leur experiences lors des formations. Ils peuvent egallement etre amener a etre surveiller la bonne utilisation du matrier lors de leurs permances. Durant ces dernieres ils peuvent apporter leur soutient aux eleves sous forme de conseilles ou en les guidant dans la bonne utilisation du materiel.
\subparagraph{Membres} Ce sont les eleves fersant partie du club, ils ont la possibiliter d'utiliser le FabLab pour leurs projet personnel dans le meusure ou ils ont suivi les formation en adequation avec le materiel qu'ils utilisent et qu'il accepte de subvenir aux couts qu'entraine leur utilisation. Ils ont la possibiliter de venir au FabLab quand ce dernier est dissponible et sous condition de presence d'un referend\footnote{Persone aillant les competences necessaires pour intervenir en cas de necessiter et pouvant apporter de l'aide en cas de besoin.}. 
\subparagraph{Eleves non-membres}
Ne fesant pas parti du club ils ont tout du moin la possibiliter de venir suivre les formations et d'utiliser le FabLab dans le cadre de leur cursus pedagogique.\\ 
En revanche, nous ne les laisseront pas en autonomie sur les machines et le permanant du moment, ou un membre, sera a leur cote pour les guider dans l'utilisation.\\
Il me semble important ici de rapeller que auqune prioriter ne sera acorder aux membre par rapport aux non-membre et meme que l'acce aux FabLab pourra meme etre reserver exclusivement aux projets pendant de periodes particuliere tel que les rushs de PPE.
\vspace{1cm}
\subparagraph{Finacement}
Comme nous consideront des projets en dehord de la mission pedagogique imediate de l'ECE il est normale que l'ecole ne soit pas tenu de les financer mais l'eleve proteur du projet. \\
Toutes fornitures\footnote{en dehord des matiers premiers necessaire a l'utilisation des machines } devra etre acheter par l'eleve. A la rare exeption de petites forniture peu cher tel que une ou deux resistance, auqun composant appartenant au stock de l'ecole ne devra etre consomer.\\
Pour l'utilisation des machines : imprimante 3d et lazer, un tarfi de location horiare sera facturer prenant en compte : le cout de la matiere premiers, l'usure de la machine et le cout de la maintenance. Le but etant le remplacement et de la matiere utiliser et le cout de la remise en etat reguliere de ces dernieres. Auqune marge ne sera tirer de cette echange au protif du club.
\subsection{La formation}
\paragraph{Problematique : }
Nous avons constater que beaucoup de groupe de projet souhaitent utiliser des technologies qu'ils ne maitrisent pas. Le cas le plus frequent etant la creation de pieces grace aux imprimantes 3D. Ceci pourrait aussi etre une bonne ocasion de sensibilier les etudiants aux limites de chaqune de ces technologies car il n'est pas rare de rencontrer des cas ou les demandes des eleves ne sont tout simplement pas realisablent.\\
Bien que certaines majeurs propose des formation sur ces technique dans les annees superieurs, un bon nombre de projets n'ont pas ecore eu l'opportunier de les faires. 
\subparagraph{Solution : }
\textbf{Formation Logicielle :} Nous penssions organiser tout le long de l'anne des petites formation afin d'enseigner les rudiments necessaire pour repondre aux demande des projets. Je pensse ici a des formations aux methodes de modelisation en vue de la realisation de pieces avec les imprimantes 3D et le lazer. \\
\textbf{Formation Machine :} En petit groupe et encadrer par un makers ces formations auront lieux directement dans le FabLab et seront des presentation d'une technologie particuliere dissponible dans le labo. Elles presenteront les bonnes pratiques, les regles de securiers et les forces/ limitation de cette derniere. Ce sont des moment privilegier pour les eleves d'ellargir leurs horizons et d'augmenter leurs conaissances techniques.
\subsection{Mise en place d'un planning}
\paragraph{Problematique :}
Grace a la presence de Daniel, ouverture du FabLab est assurer entre 8h et 16h30, quasient en continu\footnote{A l'exetion d'une eventuelle petite pause le mide.}. En revanche seul la volonter de certains eleves permet de la garder ouvert en dehord de ces horaires. Un groupe d'etudiant voulant travailler au FabLab en dehord de ces horaires n'a donc pas de moyen claire de s'assurer de son ouverture. \\
De plus, les imprimantes 3D peuvent se retrouver extremement soliciter en periode de forte afluence ( rush projet ) et il n'est pas possible de reserver actuellent de creneaux d'utilisation. 
\paragraph{Solution :}
La mise en place d'un planning claire pour la FabLab qui serait afficher sur la porte d'entrer et maintenu a jour semaine apres semaine avec les formations et les horaires d'ouvertures. Il nous permettrait aussi de repartir les permances entre les differents makers.\\
En ce qui concerne les machines, un planning de reservation pourrait etre mis a disposition: les eleves n'auraient que a venir et lancer leurs piece une fois leur creneaux venus. Ceci s'serait d'autant plus facile a mettre en oeuvre qu'il est possible de connaitre a l'avance le temps necessaire a la realisation d'une piece.
\subsection{Le conseille technique}
\paragraph{Probleme : }
Comme beaucoup d'eleves ne sont pas familier avec des technologies et souhaitent au plus vite pouvoire definir leurs besoins il viennet defois demander conseille pour savoire qu'elles outile est le plus adapter: imprimante 3D ou lazer ; gravure chimique ou a l'anglaise ?
\paragraph{Solution :} 
En creant un groups d'eleves experimenter, comprenant leurs besoin et connaissant bien les technologies a disspositions il sera plus facile pour les eleves d'obtenir conseille pour les guider.\\
Attention: il ne sagit ici que de conseille, notre mission n'est auqunement de nous impliquer dans leur projet ou de faire leur travaille a leur place.
\section{Ce qui ne releve pas de nos mission}
\subsection{Gestoin des stocks de TP}
Nous ne sommes en auqune maniere responssable du matirel que les professeurs utilise en TP d'ailleur nous ne sommes en auqune maniere tenu a aider les eleves en TP si ce dernier n'a pas un lien directe avec le FabLab. 
\subsection{Des preposer aux projets}
Notre mission ce limite a l'aide a l'utilisation des machine et a apporter des conseilles dans la realisation, nous ne sommes pas la pour realiser les projets des elves a leurs place. \\
Par exemple, si un eleve viens nous voire avec un circuit a graver il ne peut pas nous donner le circuit et attendre que nous le realisont pour lui. Il devra lui-meme faire la demarche de grave et nous ne seront present que pour l'assister en cas de difficuler.
\subsection{L'organisation de Pots ou Soirees}
Etant un club dedier a la conseption et a la creation sur un plan technique, notre manda n'englobe pas directement de dimention sociale. Nous ne jugeront pas necessaire notre implication dans l'organisation de ce type d'evenements.
\section{Adhesion}
\subsection{Cothisation}Une cotisation sera demander a tout les eleves afin de faire partie du club de X euros. Cette sera utiliser pour completter les finaces de l'association afin de finacer l'achat de nouveaux materiel et les projets du club%TODO: completer%.
\subsection{Charte de bonne conduite}
Tout nouveaux membre acepte de ce soumettre a une charte de bonne conduite, en cas de violation de cette charte une exclusion temporaire voire un blacklistage pourra etre envisager. Ceci peut sembler severre mais la dangerositer et le cout important du materiel nous interdit un certain laxisme.

\end{document}